\chapter{Results}

In this chapter, attained results during the development of this project will be
presented. First, a series of benchmarks have been used in order to
exercise key Alcaudon modules and measure how the system behaves. To conclude,
a summary of accomplished objectives is presented.

\section{Benchmarks}

Alcaudon has been tested thoroughly, both using traditional testing techniques
as well as more state-of-the-art techniques such as property based testing.
These tests can guarantee correctness in terms of behavior, but they do not
measure system correctness in terms of performance goals. Even though Alcaudon
has been designed carefully, using the adequate data structures and taking
care of performance, empiric results are needed. It is known that creating
accurate benchmarks is complex~\cite{benchbias}, there are many factors
that can lead to misleading results. Presented benchmarks are implemented using
JMH\footnote{http://openjdk.java.net/projects/code-tools/jmh/} trying to be as
accurate as possible.

\subsection{Computation execution benchmark}

This benchmark measures the number of computations per second that Alcaundon can
handle.

\FIXME{TODO -> figure}

\subsection{Stream benchmark}

This benchmark measures the number of records that a stream can handle.

\FIXME{TODO -> figure}

\subsection{Record router benchmark}

This benchmark measures the number of records that a router can route.

\FIXME{TODO -> figure}

\section{Accomplished objectives}

During the development of this project, all objectives defined at the beginning
of this document have been accomplished. Alcaudon provides all the necessary
tools to deploy distributed stream data processing pipelines. It attains this
objective by supplying an abstract model, the Computation \acs{API}, that users
implement, leveraging the all the complexities around fault-tolerant distributed
models to Alcaudon.

Regarding specific objectives, these have been accomplished as well:
%
\begin{itemize}
\item Alcaudon provides an abstraction in order to create distributed
  computations; computation \acs{API} alongside the dataflow builder.
\item The system provides exactly-once processing semantics, using at-least once
  delivery in combination with idempotent record processing. In order to achieve
  idempotence, state-of-the-art probabilistic data structures have been used as
  well as persistent actors to guarantee persistent state durability.
\item The system provides watermark based timers in order to work with
  out-of-order data. The implementation of this subsystem uses state-of-the-art
  coordination-free distributed data types in order to replicate knowledge about
  time event evolution inside the system.
\item Dataflow \acs{DAG}s are scheduled into available compute nodes using an
  state-of-the-art flexible cluster scheduler~\cite{firmament}. Different scheduling
  policies can be configured depending on the needs, where the default choice is
  biased towards co-location.
\item Several sources and sinks are provided by default such as Twitter streaming
  API, TCP Sockets or Apache Kafka. However it is possible to implement new ones
  just extending SourceFn and SinkFN interfaces.
\item Alcaudon provides an elastic distributed implementation where compute
  nodes can join the cluster in the events of burst of load. The system has been
  designed in order to be resilient and fault-tolerant. The actor model has
  helped a lot in terms of resilience due to is supervision mechanism that
  allows to isolate failure.
\item Alcaudon provides different metrics about the state of the system. It has been
  integrated with industry proven technologies in the space of monitoring such as
  Prometheus and Graphana. Using these tools it is possible to create a rich set
  of alerts and dashboards giving a good overview of how the different components
  of Alcaudon perform.
\item Alongside the previously presented results, Alcaudon is distributed using Docker
  containers and its library is available in SonaType repository. This makes almost
  trivial to deploy a full-featured Alcaudon cluster.
\end{itemize}

\section{Future work}

Given that Alcaudon has been designed in a modular way and it has many automatic
tests, including new features should effortless. In this section some future work
that could be done to the platform is presented.

\begin{itemize}
\item Improve system performance avoiding object allocations in certain
  sensitive parts of the system such as the Streams. High allocation rate
  usually hurts managed systems performance due to garbage collection. Some
  systems such as Netty\footnote{https://netty.io/} use pre-allocated memory buffers
  in order to avoid heap allocations as much as possible. This approach could be
  explored in order to improve Alcaudon performance.
\item Implement a machine learning algorithm in order to select a scheduling policy
  based on similar already run dataflow pipelines, improving the overall performance.
\item Implement a fully functional programming interface on top of the
  computation API in order to offer a more expressive API to the users.
  Providing \textit{combinators} such as map, groupBy, filter, etc.
\item Alcaudon watermark algorithm is just a minimal version. Used heuristic
  could be improved and it would be interesting to test different machine
  learning algorithms in order to predict watermarks. Another interesting
  approach to the problem that watermarks solve is to use virtual tables as
  Apache Kafka Streaming does~\cite{kafkastreams}.
\end{itemize}
