\chapter{Objectives}
\label{chap:objectives}

\drop{I}{n} this chapter, the global objectives that have motivated the project
are described. As well as the more specific goals that want to be achieved.

\section{General Objectives}

The purpose of this project is to develop a distributed, fault-tolerant and
elastic streaming processing framework aimed for unbounded datasets. Given a
user defined topology of computations, the system will place them into a cluster
of nodes optimizing the resource utilization. Alcaudon abstracts away all the
difficulties involved in programming in a distributed environment. The user just
needs to define the computations, using the Alcaudon computation Interface, data
dependencies, and the system will take care of the execution. The system should be
easily deployed into cloud platforms such as Amazon Web Services, so it can cope
with bursts of load dynamically.

\section{Specific objectives}

Given the previous general description, objectives can be categorized into the
following sub-objectives.

\subsection{Provide an abstraction to create distributed computations}
One of the primary goals of this project is to provide abstractions to create
distributed streaming programs without distributed systems expertise. To achieve
this, Alcaudon should provide a computation API that allows clients to write
their business logic without knowing any details of the underlying
infrastructure.
Alcaudon should provide means to work with persistent state in user code. A
State API is provided to work with key-value pairs.

\subsection{Develop mechanisms to ensure exactly-once processing of records}

One of the biggest problems in distributed systems is to guarantee that a record
has been delivered and processed. The system will ensure that messages are
delivered exactly-once, without any change in user code.

To achieve exactly-once delivery in a performant manner, i.e., without two-phase
commits (citation needed), Alcaudon will enforce idempotency using probabilistic
data structures and state journaling.

\subsection{Provide tools to work with out-of-order data}

Unordered data is a reality in distributed environments. Some systems enforce
monotonicity of event time using the injection time instead of the event
generation time. Alcaudon will provide tools to work with non monotonic event
times.

\subsection{Implement a cluster scheduler}

Users of Alcaudon provide a directed acyclic graph of computations. Those
tasks should be placed into the cluster available resources. Cluster scheduling
usually leads to a NP-hard problem. The system will implement a scheduler based
on heuristics to set the tasks into the computing nodes.


\subsection{Allow extensibility of sources and sinks}
Some implementations for unbounded data sources, like TCP sockets, scala collections
and twitter will be available. But users will be able to extend Alcaudon to add
custom Sources and Sinks.

\subsection{Design and implement an elastic, fault-tolerant and scalable distributed architecture}
The system should have the properties described in the reactive manifesto(citation needed):

\begin{itemize}
  \item The system should be \textit{resilient}, meaning that in the face of a failure it should keep running.
  \item The system should be \textit{elastic}, in the face of an increase of load it
    should be responsive allowing adding new resources to cope with the new
    requirements.
\end{itemize}

It should make easy to add new nodes to the cluster, so the resources available
can change depending on the needs. This design towards a more cloud oriented
architecture could facilitate offering Alcaudon as a service.

\subsection{Provide tools for observability}

The project should be designed with observability in mind. Metrics and logging
should be exposed. This allows an easier operability, giving the site
reliability engineers tools to reason about the behaviour of the system.
